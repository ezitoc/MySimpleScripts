% * Copyright 2012 Sergio García GodTIC-Templates 
% * Modificado por Martín Ludueña 
% * Modificado por Ezequiel Castillo

\usepackage[utf8x]{inputenc}
\usepackage[spanish,activeacute,es-noshorthands]{babel} % Caracteres conacentos. Modificado E. Castillo.

%\usetheme{/theme}
\usepackage{amsmath,amssymb,amsfonts,latexsym,cancel,chemarrow} %Soporte para símbolos y font matemáticos
\usepackage{thumbpdf}
\usepackage{wasysym}
\usepackage{ucs}
%\usepackage{pgf,pgfarrows,pgfnodes,pgfautomata,pgfheaps,pgfshade}
\usepackage{verbatim}
\usepackage{hyperref}
\usepackage{pifont}
\usepackage{color}
\usepackage{wrapfig}
\usepackage{graphicx}
\usepackage{textcomp}

%*****COLORS*****
\definecolor{mygreen}{RGB}{0,117,0}
\definecolor{myorange}{rgb}{1,0.5,0}
\definecolor{myviolet}{RGB}{152,152,217}


\graphicspath{{img/}} 
\DeclareGraphicsExtensions{.pdf,.png,.jpg}
\usepackage{listings} % Ambiente para incluir Codigo Fuente 

% Paquetes agregados por Martin Ludueña
\usepackage{threeparttable} %Footnotes en las tablas
\usepackage{booktabs}          
\usepackage{xmpmulti}	% para multiinclude  
\usepackage{multirow} 
\usepackage{rotating}
%*******************************************

% Paquetes agregados por Ezequiel Castillo
\usepackage{a4wide}
% \usepackage[T1]{fontenc}
% \usepackage{fouriernc}
\usepackage[scriptsize,bf]{caption} %Caption negrita y chiquita
\usepackage{tikz}
\usetikzlibrary{calc,3d}
\usepackage{gnuplot-lua-tikz}
\usepackage{grffile} %Para leer archivo .algo.pdf
\usepackage{array}
\usepackage[version=3]{mhchem}
\usepackage{booktabs}
% \usepackage{dotlessi}
% \usepackage{hyperref}
% \usepackage{sidecap} %Side Caption
% \usepackage[sc]{mathpazo}
% \usepackage{setspace}
% \onehalfspacing

% \usepackage{fancyhdr}
% \pagestyle{fancy}
% \fancyhead[LO,RE]{Izquierda}
% \fancyhead[RO,RE]{Derecha}
% \fancyfoot[CO,CE] {\thepage}
%*******************************************
% \usepackage{color}
\definecolor{lightgray}{rgb}{.9,.9,.9}
\definecolor{darkgray}{rgb}{.4,.4,.4}
\definecolor{purple}{rgb}{0.65, 0.12, 0.82}
\definecolor{ocre}{RGB}{255,255,153}

% \lstdefinelanguage{JavaScript}{
  %keywords={typeof, new, true, false, catch, function, return, null, catch, switch, var, if, in, while, do, else, case, break},
  %keywordstyle=\color{blue}\bfseries,
  %ndkeywords={class, export, boolean, throw, implements, import, this},
  %ndkeywordstyle=\color{darkgray}\bfseries,
  %identifierstyle=\color{black},
  %sensitive=false,
  %comment=[l]{//},
  %morecomment=[s]{/*}{*/},
  %commentstyle=\color{purple}\ttfamily,
  %stringstyle=\color{red}\ttfamily,
  %morestring=[b]',
  %morestring=[b]"
%}

% \lstset{tabsize=4,
  %showspaces=false,
  %showtabs=false,
  %frame=l,
  %framerule=1pt,
  %aboveskip=0.5cm,
  %framextopmargin=3pt,
  %framexbottommargin=3pt,
  %framexleftmargin=18pt,
  %framesep=.4pt,
  %rulesep=.4pt,
  %stringstyle=\ttfamily,
  %showstringspaces = false,
  %basicstyle=\footnotesize\ttfamily,
  %keywordstyle=\bfseries,
  %numbers=left,
  %numbersep=6pt,
  %numberstyle=\color[cmyk]{0.43, 0.35, 0.35,0.01}\bfseries\scriptsize\ttfamily,
  %numberfirstline = true,
  %breaklines=true,
  %stepnumber=1,
  %backgroundcolor=\color{white},
  %xleftmargin=18pt,
  %framexrightmargin=0pt,
  %xrightmargin=0pt,
  %language=JavaScript
%}

\lstdefinelanguage{Git}{
  classoffset=0,
  keywords={git},
  keywordstyle=\color{blue}\bfseries,
  classoffset=1,
  morekeywords={remote, add}, keywordstyle=\color{mygreen}\bfseries,
  classoffset=0,
  identifierstyle=\color{red},
  sensitive=false,
  stringstyle=\color{red}\ttfamily
}
\lstset{language=Git,
  backgroundcolor=\color{myviolet},
  basicstyle=\small,
  breakatwhitespace=false,
  breaklines=true,
  %commentstyle=\color{dkgreen},
  extendedchar=true,
  frame=single,
  morestring=[b]',
  moredelim=[is][\color{red}\ttfamily]{|}{|},
  framerule=0pt,
  aboveskip=1ex,
  belowskip=1ex,
  numbers=none,  % where to put the line-numbers; possible values are (none, left, right)
  %numbersep=0pt,
  numberstyle=\tiny\color{gray},
  stepnumber=2,
  %stringstyle=\color{mauve},
  tabsize=4
  %title=\lstname
}
