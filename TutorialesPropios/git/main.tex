%
%
%     File Name : main.tex
%
%       Purpose :
%
% Creation Date : 06-02-2013
%
% Last Modified : Wed 24 Apr 2013 08:48:56 PM ART
%
%    Created By :  Ezequiel Castillo
%
%
\documentclass[a4paper,12pt]{article}

\usepackage[utf8x]{inputenc}
\usepackage[spanish,activeacute,es-noshorthands]{babel} % Caracteres conacentos. Modificado E. Castillo.
\usepackage{amsmath,amssymb,amsfonts,amsthm,latexsym,cancel,chemarrow} %Soporte para símbolos y font matemáticos
\usepackage{verbatim}
\usepackage{color}
%\usepackage{wrapfig}
%\usepackage{graphicx}
\usepackage{hyperref}
%\usepackage{threeparttable} %Footnotes en las tablas
%\usepackage{booktabs}          
%\usepackage{multirow} 
%\usepackage{rotating}
%\usepackage{listings} % Ambiente para incluir Codigo Fuente 
\usepackage{a4wide}
%\usepackage{sidecap} %Side Caption
%\usepackage[scriptsize,bf]{caption} %Caption negrita y chiquita
%\usepackage{tikz}
%\usetikzlibrary{calc,3d}
%\usepackage{gnuplot-lua-tikz}
%\usepackage{grffile} %Para leer archivo .algo.pdf
%\usepackage{array}
%\usepackage[version=3]{mhchem}
%\usepackage{setspace}
%\onehalfspacing
\usepackage{parskip}
\usepackage{mdframed}
\usepackage{framed}
\usepackage{enumerate}

\usepackage{xstring}

\setlength{\parindent}{15pt}
\setlength{\parskip}{1ex plus 0.5ex minus 0.2ex}


%******MY COMMANDS*******
\DeclareMathOperator{\tr}{tr}
\newcommand{\bi}[1]{\textbf{\emph{#1}}}
\newcommand{\xvec}[0]{\textbf{x}}
\newcommand{\demo}{\noindent\textbf{Demostración: }}
\newcommand{\obse}{\noindent\textsc{Observación: }}

%******MY ENVIRONMENTS*******
\newenvironment{concept}[1][]{%
  \begin{framed}
  \IfStrEq{#1}{i}{\itshape}{}%
  }{%
  \end{framed}}%

\newmdtheoremenv[linewidth=2,skipbelow=2ex,skipabove=2ex]{theorem}{Teorema}
\usepackage{blindtext}

%*****MY COLORS*****
\definecolor{mygreen}{RGB}{0,117,0}
\definecolor{myviolet}{RGB}{200,193,249}
\definecolor{mylightyellow}{RGB}{249,248,193}


\graphicspath{{img/}} 
\DeclareGraphicsExtensions{.pdf,.png,.jpg}


%\lstdefinelanguage{Git}{
  %basicstyle=\small\ttfamily,
  %breaklines=true,
  %frame=single,
  %morestring=[b]',
  %moredelim=[is][\color{red}\ttfamily]{|}{|},
  %moredelim=[is][\color{gray}\ttfamily]{||}{||},
  %framerule=0pt,
  %rulesep=4pt,
  %framextopmargin=4pt,
  %framexbottommargin=4pt,
  %aboveskip=2ex,
  %belowskip=1ex,
  %numbers=none,  % where to put the line-numbers; possible values are (none, left, right)
  %%numbersep=0pt,
  %%numberstyle=\tiny\color{gray},
  %%stepnumber=2,
  %tabsize=4,
  %classoffset=0,
  %keywords={git},
  %keywordstyle=\color{blue}\bfseries,
  %classoffset=1,
  %morekeywords={remote, add, clone, rename, rm, fetch, pull, 
                %push, commit}, keywordstyle=\color{mygreen}\bfseries,
  %classoffset=0,
  %sensitive=false,
  %%stringstyle=\ttfamily
%}
%\lstdefinestyle{input}
  %{ 
  %backgroundcolor=\color{myviolet},
  %}
%\lstdefinestyle{output}
  %{
  %backgroundcolor=\color{mylightyellow},
  %identifierstyle=\color{gray} 
  %}

%\lstset{language=Git}


\begin{document}

\title{Quick reference guide to \texttt{Git}}
\date{}
\author{Castillo, M. Ezequiel}


\maketitle

This mini-tutorial will help you on basic \texttt{Git} commands for working
with remotes. This mini-guide can, in principle be updated periodically as I
find more useful commands.



\section{Working with \emph{your} Remotes}


\subsection{Cloning your Remotes}

First, \texttt{cd} to your project directory. Then type the following:
\begin{lstlisting}[style=input]
git clone |https://github.com/ezitoc/MySimpleScripts.git|
\end{lstlisting}
This will create a new folder at the current working directory called
\texttt{MySimpleScripts}.
This is similar to \texttt{checkout} command from Subversion.
Now if you run:
\begin{lstlisting}[style=input]
git remote -v
\end{lstlisting}
this will show:
\begin{lstlisting}[style=output]
||origin  https://github.com/ezitoc/MySimpleScripts.git||
\end{lstlisting}
which is the URL that Git has stored for the shortname to be
expanded to:

\subsection{Adding Remote Repositories}
Let's say you want to add a Remote but this time with a different shortname,
for example, \texttt{ezitoc} (be aware, you are creating a new project, if you
just want to rename, there is a special command for this).
You should type:
\begin{lstlisting}[style=input]
git remote add ezitoc |https://github.com/ezitoc/MySimpleScripts.git|
\end{lstlisting}
Now, by running:
\begin{lstlisting}[style=input]
git remote -v
\end{lstlisting}
you will get:
\begin{lstlisting}[style=output]
||ezitoc  https://github.com/ezitoc/MySimpleScripts.git||
\end{lstlisting}

\subsection{Renaming and removing a Remote}
For renaming you will have to enter the following:
\begin{lstlisting}[style=input]
git remote rename origin ezitoc
\end{lstlisting}

For removing just type:
\begin{lstlisting}[style=input]
git remote rm ezitoc
\end{lstlisting}

\subsection{Fetching and pulling}
Some definitions:
\begin{itemize}
  \item {\textbf{Fetch:}} pulls the data to your local repository but it
    doesn't automatically merge it with any of your work or modify what you're
    currently working on.
  \item {\textbf{Pull:}} if you have a branch set up, automatically fetch and
    merge a remote branch into your current branch.
\end{itemize}

To get data from your remote projects, you can run:
\begin{lstlisting}[style=input]
git fetch ezitoc
\end{lstlisting}

To pull data from your remote projects, you can run:
\begin{lstlisting}[style=input]
git pull ezitoc
\end{lstlisting}

\subsection{Creating new files}
In order to start tracking new files you can write:
\begin{lstlisting}[style=input]
git add filename
\end{lstlisting}
Once you have modified \texttt{filename} you will need to stag it, before you
push changes.

\subsection{Staging files}
To change a file that was already tracked you have to use the multi-purpose
command \texttt{add} to stag files.
\begin{lstlisting}[style=input]
git add filename
\end{lstlisting}

\subsection{Commiting}
Once you have set up your stagging area, you will need to run:
\begin{lstlisting}[style=input]
git commit
\end{lstlisting}
and you will be redirected to your editor. Before exiting you must specify a
commit message. You can also set the \texttt{-v} flag to which the diff will
be included in your commit message. Besides this, you can pass the inline flag
\texttt{-m} and a message between quotes to be included in the commit message,
which will result in no redirection to your editor.

Once you have commit your changes, you will need push the changes to the
server repository. 

\subsection{Pushing}
When you want to share your work or back up, you will have to run:
\begin{lstlisting}[style=input]
git push ezitoc master
\end{lstlisting}
where \texttt{master} is your master-branch (which is set by default when
using the \texttt{clone} command). This command works if you cloned from a
server to which you have write access and nobody have push in the meantime.


\section{Reverting changes}
\subsection{Revert a commit}
Supose you commit files you didn't intended to commit. For example say you ran
the following command:
\begin{lstlisting}[style=input]
git commit calcE.py -m 'Update calcE'
\end{lstlisting}
Then you must find the commit name with:
\begin{lstlisting}[style=input]
git log
\end{lstlisting}
We will find something like:
\begin{lstlisting}[style=output]
||commit 3fee482e587ed65acbafb46adfa953dbcd9a6e61
Author: M. Ezequiel Castillo <ezecastillo@gmail.com>
Date:   Wed Apr 24 20:09:03 2013 -0300

    Update calcE.py

commit e36950110dda5c906cf36f54626b739783db00ae
Author: M. Ezequiel Castillo <ezecastillo@gmail.com>
Date:   Wed Apr 24 20:07:07 2013 -0300

    Revert ``Update?''

origin  https://github.com/ezitoc/MySimpleScripts.git||
\end{lstlisting}

Then we can revert the commit ``\emph{Update calcE.py}'' with
\begin{lstlisting}[style=input]
git revert 3fee482e587ed65acbafb46adfa953dbcd9a6e61
\end{lstlisting}
\end{document}
