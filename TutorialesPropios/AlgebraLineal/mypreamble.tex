\usepackage[utf8x]{inputenc}
\usepackage[spanish,activeacute,es-noshorthands]{babel} % Caracteres conacentos. Modificado E. Castillo.
\usepackage{amsmath,amssymb,amsfonts,amsthm,latexsym,cancel,chemarrow} %Soporte para símbolos y font matemáticos
\usepackage{verbatim}
\usepackage{color}
%\usepackage{wrapfig}
%\usepackage{graphicx}
\usepackage{hyperref}
%\usepackage{threeparttable} %Footnotes en las tablas
%\usepackage{booktabs}          
%\usepackage{multirow} 
%\usepackage{rotating}
%\usepackage{listings} % Ambiente para incluir Codigo Fuente 
\usepackage{a4wide}
%\usepackage{sidecap} %Side Caption
%\usepackage[scriptsize,bf]{caption} %Caption negrita y chiquita
%\usepackage{tikz}
%\usetikzlibrary{calc,3d}
%\usepackage{gnuplot-lua-tikz}
%\usepackage{grffile} %Para leer archivo .algo.pdf
%\usepackage{array}
%\usepackage[version=3]{mhchem}
%\usepackage{setspace}
%\onehalfspacing
\usepackage{parskip}
\usepackage{mdframed}
\usepackage{framed}
%\usepackage{enumerate}
\usepackage[shortlabels]{enumitem}
\usepackage{fancyhdr}

\usepackage{xstring}

\setlength{\parindent}{15pt}
\setlength{\parskip}{1ex plus 0.5ex minus 0.2ex}


%******MY COMMANDS*******
\DeclareMathOperator{\tr}{tr}
\newcommand{\bi}[1]{\textbf{\emph{#1}}}
\newcommand{\demo}{\noindent\textbf{Demostración: }}
\newcommand{\obse}{\noindent\textsc{Observación: }}
\newcommand{\case}[1]{\noindent\textit{Caso #1. }}
\newcommand{\changefont}{%
  \fontsize{9}{11}\selectfont
}


%******MY OPERATORS******
\DeclareMathOperator*{\adj}{adj}

%******MY ENVIRONMENTS*******
\newenvironment{concept}[1][]{%
  \begin{framed}
  \IfStrEq{#1}{i}{\itshape}{}%
  }{%
  \end{framed}}%

\newmdtheoremenv[linewidth=2,skipbelow=2ex,skipabove=2ex]{theorem}{Teorema}[section]
\newmdtheoremenv[linewidth=2,skipbelow=2ex,skipabove=2ex]{lemma}{Lema}[section]
\usepackage{blindtext}

%*****MY COLORS*****
\definecolor{mygreen}{RGB}{0,117,0}
\definecolor{myviolet}{RGB}{200,193,249}
\definecolor{mylightyellow}{RGB}{249,248,193}


\graphicspath{{img/}} 
\DeclareGraphicsExtensions{.pdf,.png,.jpg}


%*****FANCYHDR******
\pagestyle{fancy}
\fancyhead[L]{}%\changefont \slshape \rightmark
\fancyhead[R]{\changefont \slshape \rightmark}
\fancyfoot[C]{\changefont \thepage}


%\lstdefinelanguage{Git}{
  %basicstyle=\small\ttfamily,
  %breaklines=true,
  %frame=single,
  %morestring=[b]',
  %moredelim=[is][\color{red}\ttfamily]{|}{|},
  %moredelim=[is][\color{gray}\ttfamily]{||}{||},
  %framerule=0pt,
  %rulesep=4pt,
  %framextopmargin=4pt,
  %framexbottommargin=4pt,
  %aboveskip=2ex,
  %belowskip=1ex,
  %numbers=none,  % where to put the line-numbers; possible values are (none, left, right)
  %%numbersep=0pt,
  %%numberstyle=\tiny\color{gray},
  %%stepnumber=2,
  %tabsize=4,
  %classoffset=0,
  %keywords={git},
  %keywordstyle=\color{blue}\bfseries,
  %classoffset=1,
  %morekeywords={remote, add, clone, rename, rm, fetch, pull, 
                %push, commit}, keywordstyle=\color{mygreen}\bfseries,
  %classoffset=0,
  %sensitive=false,
  %%stringstyle=\ttfamily
%}
%\lstdefinestyle{input}
  %{ 
  %backgroundcolor=\color{myviolet},
  %}
%\lstdefinestyle{output}
  %{
  %backgroundcolor=\color{mylightyellow},
  %identifierstyle=\color{gray} 
  %}

%\lstset{language=Git}
