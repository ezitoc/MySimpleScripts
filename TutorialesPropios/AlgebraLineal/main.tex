
%
%
%     File Name :
%
%       Purpose :
%
% Creation Date :
%
% Last Modified : Thu 21 Feb 2013 03:23:28 PM ART
%
%    Created By :  Ezequiel Castillo
%
%

\documentclass[a4paper,12pt]{article}

\usepackage[utf8x]{inputenc}
\usepackage[spanish,activeacute,es-noshorthands]{babel} % Caracteres conacentos. Modificado E. Castillo.
\usepackage{amsmath,amssymb,amsfonts,amsthm,latexsym,cancel,chemarrow} %Soporte para símbolos y font matemáticos
\usepackage{verbatim}
\usepackage{color}
%\usepackage{wrapfig}
%\usepackage{graphicx}
\usepackage{hyperref}
%\usepackage{threeparttable} %Footnotes en las tablas
%\usepackage{booktabs}          
%\usepackage{multirow} 
%\usepackage{rotating}
%\usepackage{listings} % Ambiente para incluir Codigo Fuente 
\usepackage{a4wide}
%\usepackage{sidecap} %Side Caption
%\usepackage[scriptsize,bf]{caption} %Caption negrita y chiquita
%\usepackage{tikz}
%\usetikzlibrary{calc,3d}
%\usepackage{gnuplot-lua-tikz}
%\usepackage{grffile} %Para leer archivo .algo.pdf
%\usepackage{array}
%\usepackage[version=3]{mhchem}
%\usepackage{setspace}
%\onehalfspacing
\usepackage{parskip}
\usepackage{mdframed}
\usepackage{framed}
\usepackage{enumerate}

\usepackage{xstring}

\setlength{\parindent}{15pt}
\setlength{\parskip}{1ex plus 0.5ex minus 0.2ex}


%******MY COMMANDS*******
\DeclareMathOperator{\tr}{tr}
\newcommand{\bi}[1]{\textbf{\emph{#1}}}
\newcommand{\xvec}[0]{\textbf{x}}
\newcommand{\demo}{\noindent\textbf{Demostración: }}
\newcommand{\obse}{\noindent\textsc{Observación: }}

%******MY ENVIRONMENTS*******
\newenvironment{concept}[1][]{%
  \begin{framed}
  \IfStrEq{#1}{i}{\itshape}{}%
  }{%
  \end{framed}}%

\newmdtheoremenv[linewidth=2,skipbelow=2ex,skipabove=2ex]{theorem}{Teorema}
\usepackage{blindtext}

%*****MY COLORS*****
\definecolor{mygreen}{RGB}{0,117,0}
\definecolor{myviolet}{RGB}{200,193,249}
\definecolor{mylightyellow}{RGB}{249,248,193}


\graphicspath{{img/}} 
\DeclareGraphicsExtensions{.pdf,.png,.jpg}


%\lstdefinelanguage{Git}{
  %basicstyle=\small\ttfamily,
  %breaklines=true,
  %frame=single,
  %morestring=[b]',
  %moredelim=[is][\color{red}\ttfamily]{|}{|},
  %moredelim=[is][\color{gray}\ttfamily]{||}{||},
  %framerule=0pt,
  %rulesep=4pt,
  %framextopmargin=4pt,
  %framexbottommargin=4pt,
  %aboveskip=2ex,
  %belowskip=1ex,
  %numbers=none,  % where to put the line-numbers; possible values are (none, left, right)
  %%numbersep=0pt,
  %%numberstyle=\tiny\color{gray},
  %%stepnumber=2,
  %tabsize=4,
  %classoffset=0,
  %keywords={git},
  %keywordstyle=\color{blue}\bfseries,
  %classoffset=1,
  %morekeywords={remote, add, clone, rename, rm, fetch, pull, 
                %push, commit}, keywordstyle=\color{mygreen}\bfseries,
  %classoffset=0,
  %sensitive=false,
  %%stringstyle=\ttfamily
%}
%\lstdefinestyle{input}
  %{ 
  %backgroundcolor=\color{myviolet},
  %}
%\lstdefinestyle{output}
  %{
  %backgroundcolor=\color{mylightyellow},
  %identifierstyle=\color{gray} 
  %}

%\lstset{language=Git}


\begin{document}

%------------------------------------------
%------------------------------------------
\section{Sistemas de ecuaciones lineales y matrices}

%------------------------------------------
\subsection{Ecuaciones lineales}

\begin{concept}[i]
  Todo sistema de ecuaciones lineales no tiene soluciones, tiene exactamente
  una solución o tiene una infinidad de soluciones.
\end{concept}

Las \textbf{\emph{operaciones elementales}} en las filas son las siguientes:

\begin{concept}
  \begin{enumerate}
    \item Multiplicar una fila por una constante diferente de cero.
    \item Intercambiar dos filas.
    \item Sumar un múltiplo de una fila a otra fila.
  \end{enumerate}
\end{concept}

%------------------------------------------
\subsection{Sistemas lineales homogéneos}
Un sistema de ecuaciones lineales es \textbf{\emph{homogéneo}} si todos los
términos constantes son cero; es decir el sistema es de la forma:
\begin{align*}
  \begin{matrix}
    a_{11}x_1 &+& a_{12}x_2 &+& \cdots &+& a_{1n}x_n &=& 0     \\
    a_{21}x_1 &+& a_{22}x_2 &+& \cdots &+& a_{2n}x_n &=& 0     \\
    \vdots    & &\vdots     & &        & &\vdots     & & \vdots\\
    a_{m1}x_1 &+& a_{m2}x_2 &+& \cdots &+& a_{mn}x_n &=& 0
  \end{matrix}
\end{align*}

Todo sistema de ecuaciones lineales homogéneo es consistente, ya que siempre
existe la \textbf{\textit{solución trivial}} (es decir, $x_1=1$, $x_2=1$,
\ldots, $x_n=1=0$).
Debido a que un sistema lineal homogéneo siempre tiene la solución trivial,
entonces para sus soluciones sólo hay dos posibilidades.

\begin{concept}
  \begin{itemize}
    \item El sistema tiene sólo la solución trivial.
    \item El sistema tiene infinidad de soluciones además de la solución
      trivial.
  \end{itemize}
\end{concept}

\begin{theorem}
  Un sistema de ecuaciones lineales homogéneo con más incógnitas que
  ecuaciones tiene infinidad de soluciones
  \label{theo:1}
\end{theorem}

%------------------------------------------
\subsection{Matrices y operaciones con matrices}

\begin{concept}[i]
  Una \textbf{matriz} es un arreglo rectangular de números. Los números en el arreglo
  se denominan \textbf{elementos} de la matriz.
\end{concept}

Una matriz general $m\times n$ puede expresarse como:
\begin{align*}
  A = \begin{bmatrix}
    a_{11}& a_{12}& \cdots& a_{1n}\\
    a_{21}& a_{22}& \cdots& a_{2n}\\
    \vdots&\vdots &       &\vdots \\
    a_{m1}& a_{m2}& \cdots& a_{mn}
  \end{bmatrix}
  = \left[ a_{ij} \right]_{m\times n} = \left[ a_{ij} \right]
\end{align*}

%------------------------------------------
\subsection{Operaciones con matrices}

\begin{concept}[i]
  Dos matrices son \textbf{iguales} si tienen el mismo tamaño y sus elementos
  correspondientes son iguales.
\end{concept}

\begin{concept}[i]
  Si $A$ y $B$ son matrices del mismo tamaño, entonces la \textbf{suma}
  $A+B$ es la matriz obtenida al sumar los elementos de $B$ con los elementos
  correspondientes de $A$. No es posible sumar o restar matrices de tamaños
  diferentes.
\end{concept}
En notación matricial:
  \begin{align*}
    (A+B)_{ij} = (A)_{ij} + (B)_{ij} = a_{ij} + b_{ij} \\
    (A-B)_{ij} = (A)_{ij} - (B)_{ij} = a_{ij} - b_{ij}
  \end{align*}

\begin{concept}[i]
  Si $A$ es cualquier matriz y $c$ es cualquier escalar, entonces el
  \textbf{producto} $cA$ es la matriz obtenida al multiplicar cada elemento de
  $A$ por $c$.
\end{concept}
En notación matricial:
\begin{equation*}
  (cA)_{ij} = c(A)_{ij} = ca_{ij}
\end{equation*}

\begin{concept}
  Si $A$ es una matriz $m\times r$ y $B$ es una matriz $r\times n$, entonces
  el \textbf{producto} $AB$ es la matriz $m\times n$ cuyos elementos se
  determinan como sigue. Para encontrar el elemento en la fila $i$ y en la
  columna $j$ de $AB$, considerar sólo la fila $i$ de la Matriz $A$ y la
  columna $j$ de la matriz $B$. Multiplicar entre si los elementos
  correspondientes del renglón y de la columna mencionados y luego sumar los
  productos resultantes.
\end{concept}
En notación matricial:
\begin{align*}
  \left[ (AB)_{ij} \right]_{m\times n} = \sum_{k=1}^r A_{ik}B_{kj}
\end{align*}
La matriz resultante será de $m\times n$.

%------------------------------------------
\subsection{Multiplicación de matrices por columnas y por renglones}

\begin{align*}
  j^{th} \textrm{ matriz columna de } AB &= A \left[ j^{th} \textrm{matriz columa de} B
  \right] \\
  i^{th}\textrm{ matriz fila de } AB &= \left[ j^{th} \textrm{ matriz columa de } A
  \right] B
\end{align*}

Si $\textbf{a}_1$, $\textbf{a}_2$, \ldots, $\textbf{a}_m$ denotan las matrices
fila de $A$ y $\textbf{b}_1$, $\textbf{b}_2$, \ldots, $\textbf{b}_n$ denotan
las matrices columna de $B$, entonces por lo establecido recién se concluye
que:

\begin{align*}
  AB =
  A\begin{bmatrix}\textbf{b}_1&\textbf{b}_2&\cdots&\textbf{b}_n \end{bmatrix}
    =
    \begin{bmatrix}A\textbf{b}_1&A\textbf{b}_2&\cdots&A\textbf{b}_n\end{bmatrix}
\end{align*}

\begin{align*}
  AB = \begin{bmatrix}
    \textbf{a}_1 \\
    \textbf{a}_2 \\
    \vdots       \\
    \textbf{a}_m
  \end{bmatrix} B = \begin{bmatrix}
    \textbf{a}_1B \\
    \textbf{a}_2B \\
    \vdots        \\
    \textbf{a}_mB
  \end{bmatrix}
\end{align*}

%------------------------------------------
\subsection{Productos de matrices como combinaciones lineales}

Sean:

\begin{align*}
  A = \begin{bmatrix}
    a_{11}& a_{12}& \cdots& a_{1n}\\
    a_{21}& a_{22}& \cdots& a_{2n}\\
    \vdots&\vdots &       &\vdots \\
    a_{m1}& a_{m2}& \cdots& a_{mn}
  \end{bmatrix} &
  & \textbf{x} = \begin{bmatrix}
    x_1    \\
    x_2    \\
    \vdots \\
    x_m
  \end{bmatrix} &
  & \textbf{b} = \begin{bmatrix}
    b_1    \\
    b_2    \\
    \vdots \\
    b_m
  \end{bmatrix} &
\end{align*}

Mediante esta elección es posible expresar al sistema de ecuaciones:
\begin{align*}
  \begin{matrix}
    a_{11}x_1 &+& a_{12}x_2 &+& \cdots &+& a_{1n}x_n &=& b_1     \\
    a_{21}x_1 &+& a_{22}x_2 &+& \cdots &+& a_{2n}x_n &=& b_2     \\
    \vdots    & &\vdots     & &        & &\vdots     & & \vdots\\
    a_{m1}x_1 &+& a_{m2}x_2 &+& \cdots &+& a_{mn}x_n &=& b_m
  \end{matrix}
\end{align*}
como:
\begin{align*}
  A\textbf{x}=\textbf{b}
\end{align*}
La matriz $A$ se denomina \textbf{\emph{matriz de coeficientes}} del sistema.


%------------------------------------------
\subsection{Transpuesta de una matriz}

\begin{concept}[i]
  Si $A$ es cualquier matriz $m\times n$, entonces la \textbf{transpuesta de
    $A$}, denotada por $A^T$, se define como la matriz $n\times m$ que se
    obtiene al intercambiar las filas y las columnas de $A$, es decir, la
    primera columna de $A^T$ es la primer fila de $A$, la segunda columna de
    $A^T$ es la segunda fila de $A$, y así sucesivamente.
\end{concept}
En notación matricial:
\begin{align*}
  \left( A^T \right)_{ij} = (A)_{ji}
\end{align*}

%------------------------------------------
\subsection{Traza de una matriz}

\begin{concept}[i]
  Si $A$ es una matriz cuadrada, entonces la \textbf{\emph{traza de A}},
  denotada por $\tr (A)$, se define como la suma de la diagonal principal de
  $A$. La traza de $A$ no está definida si $A$ no es una matriz cuadrada.
\end{concept}
En notación matricial:
\begin{align*}
  \tr (A)_{n\times n}=\sum_{i=1}^n a_{ii}
\end{align*}

%------------------------------------------
%------------------------------------------
\section{Reglas de la aritmética de matrices}

%------------------------------------------
\subsection{Propiedades de las operaciones con matrices}

Muchas de las reglas básicas de la aritmética de los números reales también se
cumplen para matrices, aunque unas cuantas no. Por ejemplo, para números
reales $a$ y $b$ siempre se cumple que $ab=ba$ (\emph{ley conmutativa de la
multiplicación}). Para matrices, sin embargo, $AB$ y $BA$ no necesariamente
son iguales.
\begin{theorem}
  Suponiendo que los tamaños de las matrices son tales que las operaciones
  indicadas se pueden efectuar, entonces son válidas las siguientes reglas de
  aritmética matricial.

  \begin{enumerate}[(a)]
    \item $A+B=B+A$ \hfill \bi{Ley conmutativa de la adición}
    \item $A+(B+C)=(A+B)+C$ \hfill \bi{Ley asociativa de la adición}
    \item $A(BC)=(AB)C$ \hfill \bi{Ley asociativa de la multiplicación}
    \item $A(B+C)=AB+AC$ \hfill \bi{Ley distributiva por la izquierda}
    \item $(B+C)A=BA+CA$ \hfill \bi{Ley distributiva por la derecha}
    \item $A(B-C)=AB-AC$
    \item $(B-C)A=BA-CA$
    \item $a(B+C)=aB+aC$
    \item $a(B-C)=aB-aC$
    \item $(a+b)C=aC+bC$
    \item $(a-b)C=aC-bC$
    \item $a(bC)=(ab)C$
    \item $a(BC)=(aB)C=B(aC)$
  \end{enumerate}
  \label{theo:aritmat}
\end{theorem}

Probaremos el inciso \emph{(d)}. Para ello es necesario probar que
$A(B+C)$ y $AB+AC$ son del mismo tamaño y que los elementos correspondientes
son iguales. Para formar $A(B+C)$, $B$ y $C$ deben ser del mismo tamaño, por
ejemplo, $n\times m$. Entonces $A$ debe tener el mismo número de columnas para
que la multiplicación pueda llevarse a cabo, digamos por ejemplo, $r\times n$
de modo que $A(B+C)$ tendrá dimensiones $r\times m$. Veamos ahora la otra
igualdad. Con estas definiciones para $A$, $B$ y $C$ se cumple que tanto
$AB$ como $AC$ tienen dimensiones de $r\times n$. Por lo tanto $A(B+C)$ y
$AB+AC$ son del mismo tamaño.
Queda entonces probar que los elementos correspondientes de $A(B+C)$ y $AB+AC$
son iguales, es decir que:
\begin{align*}
  [A(B+C)]_{ij}=[AB+AC]_{ij}
\end{align*}
para todos los valores de $i$ y $j$. Por las definiciones de adición y de
multiplicación de matrices se tiene:
\begin{align*}
  \left[ A(B+C)_{ij}
  \right]=&a_{i1}(b_{1j}+c_{1j})+a_{i2}(b_{2j}+c_{2j})+\cdots+a_{im}(b_{mj}+c_{mj}) \\
         =&a_{i1}b_{1j}+a_{i1}c_{1j}+a_{i2}b_{2j}+a_{i2}c_{2j}+\cdots+a_{im}b_{mj}+a_{im}c_{mj}
\end{align*}
Pero:
\begin{align*}
  a_{i1}b_{1j}+a_{i2}b_{2j}+\cdots+a_{im}b_{mj}=[AB]_{ij} \\
  a_{i1}c_{1j}+a_{i2}c_{2j}+\cdots+a_{im}c_{mj}=[AC]_{ij}
\end{align*}
Por lo tanto:
\begin{align*}
  [A(B+C)]_{ij}=&[AB]_{ij}+[AC]_{ij} \\
               =&[AB+AC]_{ij}
\end{align*}
Con esto queda demostrado que los elementos correspondientes de $A(B+C)$ y
$AB+AC$ son iguales.

La demostración del inciso \emph{c} es más complicada.

%------------------------------------------
\subsection{Matrices cero}

\begin{concept}
  Una matriz que tiene sus elementos iguales a cero se denomina \bi{matriz
  cero}.
\end{concept}

Como ya se sabe que algunas de las reglas de la aritmética para los números reales
no se cumplen en la aritmética matricial, es temerario asumir que todas las
propiedades del número real cero se cumplen para las matrices cero. En la
aritmética de números reales se cumple que:
\begin{itemize}
  \item Si $ab=ac$ y $a\ne 0$, entonces $b=c$ (\emph{ley de cancelación}).
  \item Si $ad=0$ entonces por lo menos uno de los factores del miembro
    izquierdo es cero.
\end{itemize}
En general los resultados correspondientes \textbf{no} son ciertos en aritmética
matricial.

%------------------------------------------
\subsection{Matrices identidad}

\begin{concept}
  Una matriz \textbf{cuadrada} que tiene unos en la diagonal principal y ceros
  fuera de ésta se denomina \bi{matriz identidad}.
\end{concept}

\end{document}
