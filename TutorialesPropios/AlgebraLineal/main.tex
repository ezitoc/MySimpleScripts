
%
%
%     File Name :
%
%       Purpose :
%
% Creation Date :
%
% Last Modified : Tue 19 Feb 2013 07:04:50 PM ART
%
%    Created By :  Ezequiel Castillo
%
%

\documentclass[a4paper,12pt]{article}

\usepackage[utf8x]{inputenc}
\usepackage[spanish,activeacute,es-noshorthands]{babel} % Caracteres conacentos. Modificado E. Castillo.
\usepackage{amsmath,amssymb,amsfonts,amsthm,latexsym,cancel,chemarrow} %Soporte para símbolos y font matemáticos
\usepackage{verbatim}
\usepackage{color}
%\usepackage{wrapfig}
%\usepackage{graphicx}
\usepackage{hyperref}
%\usepackage{threeparttable} %Footnotes en las tablas
%\usepackage{booktabs}          
%\usepackage{multirow} 
%\usepackage{rotating}
%\usepackage{listings} % Ambiente para incluir Codigo Fuente 
\usepackage{a4wide}
%\usepackage{sidecap} %Side Caption
%\usepackage[scriptsize,bf]{caption} %Caption negrita y chiquita
%\usepackage{tikz}
%\usetikzlibrary{calc,3d}
%\usepackage{gnuplot-lua-tikz}
%\usepackage{grffile} %Para leer archivo .algo.pdf
%\usepackage{array}
%\usepackage[version=3]{mhchem}
%\usepackage{setspace}
%\onehalfspacing
\usepackage{parskip}
\usepackage{mdframed}
\usepackage{framed}
\usepackage{enumerate}

\usepackage{xstring}

\setlength{\parindent}{15pt}
\setlength{\parskip}{1ex plus 0.5ex minus 0.2ex}


%******MY COMMANDS*******
\DeclareMathOperator{\tr}{tr}
\newcommand{\bi}[1]{\textbf{\emph{#1}}}
\newcommand{\xvec}[0]{\textbf{x}}
\newcommand{\demo}{\noindent\textbf{Demostración: }}
\newcommand{\obse}{\noindent\textsc{Observación: }}
\newcommand{\case}[1]{\noindent\textit{Caso #1. }}


%******MY OPERATORS******
\DeclareMathOperator*{\adj}{adj}

%******MY ENVIRONMENTS*******
\newenvironment{concept}[1][]{%
  \begin{framed}
  \IfStrEq{#1}{i}{\itshape}{}%
  }{%
  \end{framed}}%

\newmdtheoremenv[linewidth=2,skipbelow=2ex,skipabove=2ex]{theorem}{Teorema}
\newmdtheoremenv[linewidth=2,skipbelow=2ex,skipabove=2ex]{lemma}{Lema}
\usepackage{blindtext}

%*****MY COLORS*****
\definecolor{mygreen}{RGB}{0,117,0}
\definecolor{myviolet}{RGB}{200,193,249}
\definecolor{mylightyellow}{RGB}{249,248,193}


\graphicspath{{img/}} 
\DeclareGraphicsExtensions{.pdf,.png,.jpg}


%\lstdefinelanguage{Git}{
  %basicstyle=\small\ttfamily,
  %breaklines=true,
  %frame=single,
  %morestring=[b]',
  %moredelim=[is][\color{red}\ttfamily]{|}{|},
  %moredelim=[is][\color{gray}\ttfamily]{||}{||},
  %framerule=0pt,
  %rulesep=4pt,
  %framextopmargin=4pt,
  %framexbottommargin=4pt,
  %aboveskip=2ex,
  %belowskip=1ex,
  %numbers=none,  % where to put the line-numbers; possible values are (none, left, right)
  %%numbersep=0pt,
  %%numberstyle=\tiny\color{gray},
  %%stepnumber=2,
  %tabsize=4,
  %classoffset=0,
  %keywords={git},
  %keywordstyle=\color{blue}\bfseries,
  %classoffset=1,
  %morekeywords={remote, add, clone, rename, rm, fetch, pull, 
                %push, commit}, keywordstyle=\color{mygreen}\bfseries,
  %classoffset=0,
  %sensitive=false,
  %%stringstyle=\ttfamily
%}
%\lstdefinestyle{input}
  %{ 
  %backgroundcolor=\color{myviolet},
  %}
%\lstdefinestyle{output}
  %{
  %backgroundcolor=\color{mylightyellow},
  %identifierstyle=\color{gray} 
  %}

%\lstset{language=Git}


\begin{document}

%------------------------------------------
%------------------------------------------
\section{Sistemas de ecuaciones lineales y matrices}

%------------------------------------------
\subsection{Ecuaciones lineales}

\begin{concept}[i]
  Todo sistema de ecuaciones lineales no tiene soluciones, tiene exactamente
  una solución o tiene una infinidad de soluciones.
\end{concept}

Las \textbf{\emph{operaciones elementales}} en las filas son las siguientes:

\begin{concept}
  \begin{enumerate}
    \item Multiplicar una fila por una constante diferente de cero.
    \item Intercambiar dos filas.
    \item Sumar un múltiplo de una fila a otra fila.
  \end{enumerate}
\end{concept}

%------------------------------------------
\subsection{Sistemas lineales homogéneos}
Un sistema de ecuaciones lineales es \textbf{\emph{homogéneo}} si todos los
términos constantes son cero; es decir el sistema es de la forma:
\begin{align*}
  \begin{matrix}
    a_{11}x_1 &+& a_{12}x_2 &+& \cdots &+& a_{1n}x_n &=& 0     \\
    a_{21}x_1 &+& a_{22}x_2 &+& \cdots &+& a_{2n}x_n &=& 0     \\
    \vdots    & &\vdots     & &        & &\vdots     & & \vdots\\
    a_{m1}x_1 &+& a_{m2}x_2 &+& \cdots &+& a_{mn}x_n &=& 0
  \end{matrix}
\end{align*}

Todo sistema de ecuaciones lineales homogéneo es consistente, ya que siempre
existe la \textbf{\textit{solución trivial}} (es decir, $x_1=1$, $x_2=1$,
\ldots, $x_n=1=0$).
Debido a que un sistema lineal homogéneo siempre tiene la solución trivial,
entonces para sus soluciones sólo hay dos posibilidades.

\begin{concept}
  \begin{itemize}
    \item El sistema tiene sólo la solución trivial.
    \item El sistema tiene infinidad de soluciones además de la solución
      trivial.
  \end{itemize}
\end{concept}

\begin{theorem}
  Un sistema de ecuaciones lineales homogéneo con más incógnitas que
  ecuaciones tiene infinidad de soluciones
  \label{theo:1}
\end{theorem}

%------------------------------------------
\subsection{Matrices y operaciones con matrices}

\begin{concept}[i]
  Una \textbf{matriz} es un arreglo rectangular de números. Los números en el arreglo
  se denominan \textbf{elementos} de la matriz.
\end{concept}

Una matriz general $m\times n$ puede expresarse como:
\begin{align*}
  A = \begin{bmatrix}
    a_{11}& a_{12}& \cdots& a_{1n}\\
    a_{21}& a_{22}& \cdots& a_{2n}\\
    \vdots&\vdots &       &\vdots \\
    a_{m1}& a_{m2}& \cdots& a_{mn}
  \end{bmatrix}
  = \left[ a_{ij} \right]_{m\times n} = \left[ a_{ij} \right]
\end{align*}

%------------------------------------------
\subsection{Operaciones con matrices}

\begin{concept}[i]
  Dos matrices son \textbf{iguales} si tienen el mismo tamaño y sus elementos
  correspondientes son iguales.
\end{concept}

\begin{concept}[i]
  Si $A$ y $B$ son matrices del mismo tamaño, entonces la \textbf{suma}
  $A+B$ es la matriz obtenida al sumar los elementos de $B$ con los elementos
  correspondientes de $A$. No es posible sumar o restar matrices de tamaños
  diferentes.
\end{concept}
En notación matricial:
  \begin{align*}
    (A+B)_{ij} = (A)_{ij} + (B)_{ij} = a_{ij} + b_{ij} \\
    (A-B)_{ij} = (A)_{ij} - (B)_{ij} = a_{ij} - b_{ij}
  \end{align*}

\begin{concept}[i]
  Si $A$ es cualquier matriz y $c$ es cualquier escalar, entonces el
  \textbf{producto} $cA$ es la matriz obtenida al multiplicar cada elemento de
  $A$ por $c$.
\end{concept}
En notación matricial:
\begin{equation*}
  (cA)_{ij} = c(A)_{ij} = ca_{ij}
\end{equation*}

\begin{concept}
  Si $A$ es una matriz $m\times r$ y $B$ es una matriz $r\times n$, entonces
  el \textbf{producto} $AB$ es la matriz $m\times n$ cuyos elementos se
  determinan como sigue. Para encontrar el elemento en la fila $i$ y en la
  columna $j$ de $AB$, considerar sólo la fila $i$ de la Matriz $A$ y la
  columna $j$ de la matriz $B$. Multiplicar entre si los elementos
  correspondientes del renglón y de la columna mencionados y luego sumar los
  productos resultantes.
\end{concept}
En notación matricial:
\begin{align*}
  \left[ (AB)_{ij} \right]_{m\times n} = \sum_{k=1}^r A_{ik}B_{kj}
\end{align*}
La matriz resultante será de $m\times n$.

%------------------------------------------
\subsection{Multiplicación de matrices por columnas y por renglones}

\begin{align*}
  j^{th} \textrm{ matriz columna de } AB &= A \left[ j^{th} \textrm{matriz columa de} B
  \right] \\
  i^{th}\textrm{ matriz fila de } AB &= \left[ j^{th} \textrm{ matriz columa de } A
  \right] B
\end{align*}

Si $\textbf{a}_1$, $\textbf{a}_2$, \ldots, $\textbf{a}_m$ denotan las matrices
fila de $A$ y $\textbf{b}_1$, $\textbf{b}_2$, \ldots, $\textbf{b}_n$ denotan
las matrices columna de $B$, entonces por lo establecido recién se concluye
que:

\begin{align*}
  AB =
  A\begin{bmatrix}\textbf{b}_1&\textbf{b}_2&\cdots&\textbf{b}_n \end{bmatrix}
    =
    \begin{bmatrix}A\textbf{b}_1&A\textbf{b}_2&\cdots&A\textbf{b}_n\end{bmatrix}
\end{align*}

\begin{align*}
  AB = \begin{bmatrix}
    \textbf{a}_1 \\
    \textbf{a}_2 \\
    \vdots       \\
    \textbf{a}_m
  \end{bmatrix} B = \begin{bmatrix}
    \textbf{a}_1B \\
    \textbf{a}_2B \\
    \vdots        \\
    \textbf{a}_mB
  \end{bmatrix}
\end{align*}

%------------------------------------------
\subsection{Productos de matrices como combinaciones lineales}

Sean:

\begin{align*}
  A = \begin{bmatrix}
    a_{11}& a_{12}& \cdots& a_{1n}\\
    a_{21}& a_{22}& \cdots& a_{2n}\\
    \vdots&\vdots &       &\vdots \\
    a_{m1}& a_{m2}& \cdots& a_{mn}
  \end{bmatrix} &
  & \textbf{x} = \begin{bmatrix}
    x_1    \\
    x_2    \\
    \vdots \\
    x_m
  \end{bmatrix} &
  & \textbf{b} = \begin{bmatrix}
    b_1    \\
    b_2    \\
    \vdots \\
    b_m
  \end{bmatrix} &
\end{align*}

Mediante esta elección es posible expresar al sistema de ecuaciones:
\begin{align*}
  \begin{matrix}
    a_{11}x_1 &+& a_{12}x_2 &+& \cdots &+& a_{1n}x_n &=& b_1     \\
    a_{21}x_1 &+& a_{22}x_2 &+& \cdots &+& a_{2n}x_n &=& b_2     \\
    \vdots    & &\vdots     & &        & &\vdots     & & \vdots\\
    a_{m1}x_1 &+& a_{m2}x_2 &+& \cdots &+& a_{mn}x_n &=& b_m
  \end{matrix}
\end{align*}
como:
\begin{align*}
  A\textbf{x}=\textbf{b}
\end{align*}


\end{document}
